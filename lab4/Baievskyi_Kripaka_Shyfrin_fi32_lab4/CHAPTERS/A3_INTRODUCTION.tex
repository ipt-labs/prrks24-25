%!TEX root = ../thesis.tex
% створюємо вступ
\section{Атуальність дослідження}

 

На вiдмiну вiд iнших складових частин, саме протоколи реєстрацiї є найменш стандартизованими. З огляду на використання багатьох пристроїв та зручностi перемикатися мiж ними, реєстрацiя декiлькох пристроїв стала важливою функцiєю для користувачiв. Актуальним питанням залишається безпека реєстрацiї та користування кiлькома супутнiми пристроями.

\subsection{Постановка задачі та варіант}
\begin{tabularx}{\textwidth}{X|X}
	\textbf{Треба виконати} & \textbf{Зроблено} \\
    Провести налаштування обраної системи та виконати тестові операції в системі. & \checkmark \\
    Порівняти особливості розгортання різних систем криптовалют із системою Ethereum. & \checkmark \\
\end{tabularx}

\section{Хід роботи/Опис труднощів}
    На початку роботи над практикумом вибрали гуртом 2 варіант. Згідно вибраного варіанту у даній роботі буде продемонстровано спробу запуску, налаштування системи Ethereum та виконання тестових операцій в ній. Також наведено короткий порівняльний аналіз даної системи з іншими.