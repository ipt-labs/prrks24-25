\documentclass{bachelor_report}

% Додаткові пакети вносіть у цей файл
\input{01_packages}
\addbibresource{../BIB/Resources.bib}

% Додаткові визначення та перевизначення команд вносіть у цей файл
\input{02_redefinitions}

% Відомості про автора роботи
\input{03_data}

% Починаємо верстку документа
\begin{document}

\setfontsize{14}

% Створюємо титульну сторінку
\input{../CHAPTERS/a1_title}

%% Створюємо зміст    % -- розкоментуйте, якщо зміст вам потрібен
%\pagenumbering{gobble}
\tableofcontents
\cleardoublepage
%\pagenumbering{arabic}

\setcounter{page}{2}    %!!! -- продумати, як автоматизувати номер сторінки

%% Якщо ви використовуєте зміст, то прослідкуйте, щоб номер сторінки 
%% співпадав із справжнім!

% Створюємо перелік умовних позначень, скорочень і термінів
% Якщо цей розділ вам не потрібен, просто закоментуйте два наступних рядка
% \shortings
% \input{../CHAPTERS/a2_shortings}

% Створюємо вступ
\intro
\input{../CHAPTERS/a3_introduction}

% Додаємо глави
% Якщо ваша робота містить менше або більше глав - модифікуйте наступні 
% рядки відповідним чином
\input{../CHAPTERS/c1_chapter_01}
% \input{../CHAPTERS/c2_chapter_02}

% Створюємо висновки
\conclusions
\input{../CHAPTERS/w1_conclusions}

% Додаємо бібліографію
% Якщо ви володієте магією bibtex-у, використовуйте її та модифікуйте файл 
% з бібліографією відповідним чином
% \bibliographylist
% \printbibliography[heading=none]

% Створюємо додатки (дивись у файли додатків для необхідних пояснень)
% Якщо ви маєте меншу або більшу кількість додатків, модифікуйте наступні 
% рядки відповідним чином
% Якщо ви не маєте додатків, просто закоментуйте наступні рядки
%\input{../CHAPTERS/z1_appendix_A}


% Нарешті
\end{document}