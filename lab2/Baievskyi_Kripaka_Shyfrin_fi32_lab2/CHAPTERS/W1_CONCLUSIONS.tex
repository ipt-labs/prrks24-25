%!TEX root = ../thesis.tex
% створюємо Висновки до всієї роботи

У ході виконання дослідницької роботи було проведено аналіз реалізації протоколів IPSec, їх архітектури, функціональних можливостей та взаємодії з мережею. IPSec зарекомендував себе як надійний механізм захисту даних на мережевому рівні моделі OSI, забезпечуючи автентифікацію, цілісність і конфіденційність. Основною перевагою є його незалежність від протоколів вищих рівнів, що дозволяє захищати будь-який IP-трафік.

Архітектура IPSec представлена багаторівневою структурою, яка включає протоколи AH, ESP та ISAKMP. AH забезпечує автентифікацію і цілісність, ESP додає конфіденційність через шифрування, а ISAKMP керує параметрами безпеки. Домен інтерпретації (DOI) забезпечує стандартизацію і сумісність між різними реалізаціями.

Особливу увагу приділено концепції безпечних асоціацій (SA), які визначають параметри безпеки та управляються через бази SPD і SAD. Протоколи AH і ESP підтримують транспортний і тунельний режими, що дозволяє адаптувати їх до різних мережевих сценаріїв.

Було досліджено структуру заголовків протоколів AH і ESP, процес обробки пакетів, а також підтримувані криптографічні алгоритми. Завдяки модульності архітектури IPSec забезпечує гнучкість і високий рівень захисту IP-трафіку.

IPSec ефективно використовується у схемах хост-хост, шлюз-шлюз, хост-шлюз та для побудови VPN-тунелів, що гарантують безпечну передачу даних між віддаленими мережами або пристроями.

Таким чином, IPSec залишається одним із ключових рішень для захисту даних у сучасних мережах завдяки своїй універсальності, модульності та підтримці сучасних криптографічних механізмів.


