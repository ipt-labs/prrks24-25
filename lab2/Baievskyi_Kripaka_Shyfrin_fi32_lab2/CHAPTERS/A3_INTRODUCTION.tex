%!TEX root = ../thesis.tex
% створюємо вступ
\section{Мета практикуму}

Дослідження особливостей реалізації криптографічних механізмів протоколів IPSec.

\subsection{Постановка задачі}
\hspace{-1cm}
\begin{tabularx}{\textwidth}{p{0.8\textwidth}|>{\centering\arraybackslash}X}
	\textbf{Треба виконати} & \textbf{Зроблено} \\
    Провести дослідницьку роботу з метою аналізу особливостей реалізації криптографічних механізмів протоколів IPSec. & \checkmark \\
    Описати основне призначення протоколів IPSec, їх місце в мережевій моделі OSI та взаємодію зі стеком протоколів TCP/IP та ін. & \checkmark \\
    Дослідити архітектуру стеку протоколів IPSec. & \checkmark \\
    Описати призначення, особливості та відмінності криптографічних механізмів протоколів AH, ESP, ISAKMP, IKE, IKEv2, KINK та ін. & \checkmark \\
    Проаналізувати концепцію безпечних асоціацій (SA), її особливості та бази даних SPD і SAD (їх призначення та способи заповнення і використання). & \checkmark \\
    Дослідити детально особливості структури заголовків протоколів AH і ESP в тунельному та транспортному режимах з повним описанням їх полів та можливих значень. & \checkmark \\
    Визначити особливості обробки вхідних та вихідних IPSec-пакетів для кожного з протоколів та режимів. & \checkmark \\
    Дослідити нижній рівень архітектури стеку протоколів IPSec – домен інтерпретації DOI. & \checkmark \\
    Визначити зареєстровані алгоритми автентифікації, шифрування, геш-функцій та ін. криптографічних алгоритмів для стеку протоколів IPSec. & \checkmark \\
    Визначити та описати особливості основних схем застосування протоколів IPSec: хост-хост, шлюз-шлюз та хост-шлюз. А також використання протоколів IPSec для побудови VPN-тунелів. & \checkmark \\
\end{tabularx}